%!TeX spellcheck = en_US
\documentclass[a4paper]{report}
\usepackage[T1]{fontenc}
\usepackage[utf8]{inputenc}
\usepackage[english]{babel}
\usepackage{geometry}
\usepackage{graphicx}
\usepackage{subfig}
\usepackage{lipsum}
\usepackage{verbatim}
\usepackage[table,xcdraw]{xcolor}
\geometry{a4paper,top=2.5cm,bottom=2.5cm,left=3cm,right=3cm,%
	heightrounded,bindingoffset=5mm}

\usepackage{color}
\usepackage{listings}
\usepackage{xcolor}

\colorlet{punct}{red!60!black}
\definecolor{background}{HTML}{EEEEEE}
\definecolor{delim}{RGB}{20,105,176}
\colorlet{numb}{magenta!60!black}


\lstdefinelanguage{json}{
	basicstyle=\normalfont\ttfamily,
	numbers=left,
	numberstyle=\scriptsize,
	stepnumber=1,
	numbersep=8pt,
	showstringspaces=false,
	breaklines=true,
	frame=lines,
	backgroundcolor=\color{background},
	literate=
	*{:}{{{\color{punct}{:}}}}{1}
	{,}{{{\color{punct}{,}}}}{1}
	{\{}{{{\color{delim}{\{}}}}{1}
	{\}}{{{\color{delim}{\}}}}}{1}
	{[}{{{\color{delim}{[}}}}{1}
	{]}{{{\color{delim}{]}}}}{1},
}



\lstdefinelanguage{code}{ 
	backgroundcolor=\color{white},   % choose the background color; you must add \usepackage{color} or \usepackage{xcolor}; should come as last argument
	basicstyle=\footnotesize,        % the size of the fonts that are used for the code
	breakatwhitespace=false,         % sets if automatic breaks should only happen at whitespace
	breaklines=true,                 % sets automatic line breaking
	captionpos=b,                    % sets the caption-position to bottom
	commentstyle=\color{black},    % comment style
	deletekeywords={...},            % if you want to delete keywords from the given language
	escapeinside={\%*}{*)},          % if you want to add LaTeX within your code
	extendedchars=true,              % lets you use non-ASCII characters; for 8-bits encodings only, does not work with UTF-8
	firstnumber=1,                % start line enumeration with line 1
	frame=single,	                   % adds a frame around the code
	keepspaces=true,                 % keeps spaces in text, useful for keeping indentation of code (possibly needs columns=flexible)
	keywordstyle=\color{black},       % keyword style
	language=Octave,                 % the language of the code
	morekeywords={*,...},            % if you want to add more keywords to the set
	numbers=left,                    % where to put the line-numbers; possible values are (none, left, right)
	numbersep=5pt,                   % how far the line-numbers are from the code
	numberstyle=\tiny\color{black}, % the style that is used for the line-numbers
	rulecolor=\color{black},         % if not set, the frame-color may be changed on line-breaks within not-black text (e.g. comments (green here))
	showspaces=false,                % show spaces everywhere adding particular underscores; it overrides 'showstringspaces'
	showstringspaces=false,          % underline spaces within strings only
	showtabs=false,                  % show tabs within strings adding particular underscores
	stepnumber=1,                    % the step between two line-numbers. If it's 1, each line will be numbered
	stringstyle=\color{black},     % string literal style
	tabsize=2,	                   % sets default tabsize to 2 spaces
	literate=
	*{:}{{{\color{black}{:}}}}{1}
	{,}{{{\color{black}{,}}}}{1}
	{\{}{{{\color{black}{\{}}}}{1}
	{\}}{{{\color{black}{\}}}}}{1}
	{[}{{{\color{black}{[}}}}{1}
	{]}{{{\color{black}{]}}}}{1},
}


\lstset{ %
	language=C++,                % choose the language of the code
	basicstyle=\footnotesize,       % the size of the fonts that are used for the code
	numbers=left,                   % where to put the line-numbers
	numberstyle=\footnotesize,      % the size of the fonts that are used for the line-numbers
	stepnumber=1,                   % the step between two line-numbers. If it is 1 each line will be numbered
	numbersep=5pt,                  % how far the line-numbers are from the code
	backgroundcolor=\color{white},  % choose the background color. You must add \usepackage{color}
	showspaces=false,               % show spaces adding particular underscores
	showstringspaces=false,         % underline spaces within strings
	showtabs=false,                 % show tabs within strings adding particular underscores
	frame=single,           % adds a frame around the code
	tabsize=2,          % sets default tabsize to 2 spaces
	captionpos=b,           % sets the caption-position to bottom
	breaklines=true,        % sets automatic line breaking
	breakatwhitespace=false,    % sets if automatic breaks should only happen at whitespace
	escapeinside={\%*}{*)}          % if you want to add a comment within your code
}



\newcommand{\HRule}{\rule{\linewidth}{0.5mm}}

\begin{document}
	\begin{titlepage}
		\begin{center}
			
			% Top 
			\includegraphics[width=0.45\textwidth]{img/unipi.png}~\\[2.5cm]
			
			
			% Title
			\HRule \\[0.4cm]
			{ \LARGE 
				\textbf{SpamDetector}\\[0.4cm]
				{Project for Data Mining and Machine Learning}\\[0.4cm]
			}
			\HRule \\[1.5cm]
			
			
			
			% Author
			{ \large
				Francesco Campilongo \\[0.1cm]
			}
			
			\vfill
			
		\end{center}
	\end{titlepage}

\tableofcontents
\chapter{Introduction}
The purpose of this documentation is to prove the machine learning algorithms utility for the detection of spam, useless messages, into dataset of E-Mails and SMS.

\noindent The project is carried out using python.
\section{Datasets}
The datasets used for this project are two:
\begin{itemize}
	\item E-mail dataset
	\item SMS dataset
\end{itemize}
Those two datasets where found on Kaggle.com both of them in a .csv format.
\section{Data Preprocessing}
The datasets found were already pretty good for the type of result this project wants to achieve.

\noindent The E-mail dataset is characterize from three column a "label", "text" and "label\_num", the second column contains the actual message, the first and third column are the same, but in the first one there are the actual words "ham", for the messages which are not classified as spam, and "spam",  instead in the third column there is there are "0" for the ham messages and "1" for the spam messages;

\noindent The first column has been discharged since not useful for the execution of the classification algorithm.

\noindent The SMS file had need a little more work to obtain a dataset usable, because the it has five columns named "v1", "v2", "", "" and "", the first contains the words "ham", for the messages which are not classified as spam and "spam"; the second column contains the actual messages and the last three columns are basically blank, so in order to obtain a good dataset to work with the last three columns has been discharged the first two are taken in consideration, but in the first one the "ham" word is changed with a "0" and the "spam" word with a "1", in order to have the same kind of datasets between the E-mail and the SMSs.

\chapter{Design}
The application design is pretty basic, it is written in python (version 3) does not involve any classes, just the creation of some functions in order to keep the code clean and more understandable.
\section{Implementation}
\subsection{Libraries}
The libraries involved into this application are two:
\begin{itemize}
	\item Pandas
	\item Sklearn
\end{itemize}
Pandas gives the possibility to use a dataset in .csv (and many others) from the disk, implementing it on a dataframe into the main memory to use all the machine learning algorithm that are imported from Sklearn.
\subsection{Functions}
The functions are meant to implement all the algorithm used, which are the following:
\begin{itemize}
	\item Support Vector Machine
	\item Multinomial Näive Bayes
	\item K Nearest Neighbor
	\item Random Forest
	\item AdaBoost
\end{itemize}

\noindent These functions are written to use them with different datasets, the SMS dataset and the E-Mail dataset as already mention earlier.

\noindent There is also another function that is called to show the result of all the algorithms.
\newpage
\subsection{Main code}
This section is used to show some insight on the main code.

\noindent Here will be shown a bit of the dataset cleaning and processing that has been made, and described into the Data Preprocessing chapter. 

\lstset{ language=json}
\begin{lstlisting}
	df2 = pd.read_csv("sms_spam.csv", encoding = "ISO-8859-1")
	dfsp = df2[['v1', 'v2']]
	dfsp.loc[dfsp["v1"] == 'ham', "Category"] = 0
	dfsp.loc[dfsp["v1"] == 'spam', "Category"] = 1
	dfsp = dfsp.rename(columns={'v2': 'Content'})
	dfs = dfsp[['Content', 'Category']]
\end{lstlisting}

\noindent The following code will give a look at one of the function, the one that implement the Support Vector Machine. Since these functions are quite similar there is no need to show them all here.

\lstset{ language=json}
\begin{lstlisting}
	def SVM(x_train, y_trainFeat, y_testFeat):
		classifier = LinearSVC()
		classifier.fit(y_trainFeat, x_train)
		predRes = classifier.predict(y_testFeat)
		return predRes
\end{lstlisting}

\noindent The following code lines will show the functions calls, also in this case in order to avoid repetitions only few functions call will be shown.

\lstset{ language=json}
\begin{lstlisting}
	# KNN
	predResMailKNN = KNN(x_train, y_trainFeat, y_testFeat)
	# Metrics and results
	print("\tK Nearest Neighbors RESULTS")
	print("Neighbors Number: 1")
	show_res(actual_x, predResMailKNN)
	
	# RF
	predResMailRF = RF(x_train, y_trainFeat, y_testFeat)
	# Metrics and results
	print("\tRandom Forest RESULTS")
	show_res(actual_x, predResMailRF)
\end{lstlisting}
\chapter{Results}
The results are as expected very good .....scrivere roba riguardante il motivo per il quale i risultati sono expected magari elencando alcune peculiarità degli algoritmi 

\noindent The metrics that shows the results goodness are: Accuracy, F Score and Confusion Matrix; Chosen in order to give a deep look at the algorithms behavior.

\noindent Here the results for what concern the E-mail dataset:

\begin{lstlisting}
	/-------------------SpamDetector for e-Mail-------------------/
	Support Vector Machine results
	Accuracy Score: 98.6473
	F Score:  98.3941
	Confusion Matrix:
	[[716  13]
	[  1 305]]
	
	Multinomial Naive Bayes results
	Accuracy Score: 91.3043
	F Score:  88.5304
	Confusion Matrix:
	[[727   2]
	[ 88 218]]
	
	K Nearest Neighbors results
	Neighbors Number: 1
	Accuracy Score: 95.8454
	F Score:  94.9084
	Confusion Matrix:
	[[718  11]
	[ 32 274]]
	
	Random Forest results
	Accuracy Score: 95.4589
	F Score:  94.5432
	Confusion Matrix:
	[[706  23]
	[ 24 282]]
	
	Adaboost results
	Estimators Number: 100
	Accuracy Score: 96.9082
	F Score:  96.3159
	Confusion Matrix:
	[[709  20]
	[ 12 294]]
\end{lstlisting}

\newpage
\noindent Here the results for what concern the SMS dataset:

\begin{lstlisting}
	/---------------------SpamDetector for SMS--------------------/
	Support Vector Machine results
	Accuracy Score: 98.5650
	F Score:  96.8657
	Confusion Matrix:
	[[960   0]
	[ 16 139]]
	
	Multinomial Naive Bayes results
	Accuracy Score: 97.1300
	F Score:  93.4249
	Confusion Matrix:
	[[960   0]
	[ 32 123]]
	
	K Nearest Neighbors results
	Neighbors Number: 1
	Accuracy Score: 94.7982
	F Score:  87.0259
	Confusion Matrix:
	[[960   0]
	[ 58  97]]
	
	Random Forest results
	Accuracy Score: 97.4888
	F Score:  94.3510
	Confusion Matrix:
	[[959   1]
	[ 27 128]]
	
	Adaboost results
	Estimators Number: 100
	Accuracy Score: 97.7578
	F Score:  95.0885
	Confusion Matrix:
	[[956   4]
	[ 21 134]]
\end{lstlisting}
\end{document}