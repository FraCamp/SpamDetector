%!TeX spellcheck = en_US
\documentclass[a4paper]{report}
\usepackage[T1]{fontenc}
\usepackage[utf8]{inputenc}
\usepackage[english]{babel}
\usepackage{geometry}
\usepackage{graphicx}
\usepackage{subfig}
\usepackage{lipsum}
\usepackage{verbatim}
\usepackage[table,xcdraw]{xcolor}
\geometry{a4paper,top=2.5cm,bottom=2.5cm,left=3cm,right=3cm,%
	heightrounded,bindingoffset=5mm}

\usepackage{color}
\usepackage{listings}
\usepackage{xcolor}

\colorlet{punct}{red!60!black}
\definecolor{background}{HTML}{EEEEEE}
\definecolor{delim}{RGB}{20,105,176}
\colorlet{numb}{magenta!60!black}


\lstdefinelanguage{json}{
	basicstyle=\normalfont\ttfamily,
	numbers=left,
	numberstyle=\scriptsize,
	stepnumber=1,
	numbersep=8pt,
	showstringspaces=false,
	breaklines=true,
	frame=lines,
	backgroundcolor=\color{background},
	literate=
	*{:}{{{\color{punct}{:}}}}{1}
	{,}{{{\color{punct}{,}}}}{1}
	{\{}{{{\color{delim}{\{}}}}{1}
	{\}}{{{\color{delim}{\}}}}}{1}
	{[}{{{\color{delim}{[}}}}{1}
	{]}{{{\color{delim}{]}}}}{1},
}



\lstdefinelanguage{code}{ 
	backgroundcolor=\color{white},   % choose the background color; you must add \usepackage{color} or \usepackage{xcolor}; should come as last argument
	basicstyle=\footnotesize,        % the size of the fonts that are used for the code
	breakatwhitespace=false,         % sets if automatic breaks should only happen at whitespace
	breaklines=true,                 % sets automatic line breaking
	captionpos=b,                    % sets the caption-position to bottom
	commentstyle=\color{black},    % comment style
	deletekeywords={...},            % if you want to delete keywords from the given language
	escapeinside={\%*}{*)},          % if you want to add LaTeX within your code
	extendedchars=true,              % lets you use non-ASCII characters; for 8-bits encodings only, does not work with UTF-8
	firstnumber=1,                % start line enumeration with line 1
	frame=single,	                   % adds a frame around the code
	keepspaces=true,                 % keeps spaces in text, useful for keeping indentation of code (possibly needs columns=flexible)
	keywordstyle=\color{black},       % keyword style
	language=Octave,                 % the language of the code
	morekeywords={*,...},            % if you want to add more keywords to the set
	numbers=left,                    % where to put the line-numbers; possible values are (none, left, right)
	numbersep=5pt,                   % how far the line-numbers are from the code
	numberstyle=\tiny\color{black}, % the style that is used for the line-numbers
	rulecolor=\color{black},         % if not set, the frame-color may be changed on line-breaks within not-black text (e.g. comments (green here))
	showspaces=false,                % show spaces everywhere adding particular underscores; it overrides 'showstringspaces'
	showstringspaces=false,          % underline spaces within strings only
	showtabs=false,                  % show tabs within strings adding particular underscores
	stepnumber=1,                    % the step between two line-numbers. If it's 1, each line will be numbered
	stringstyle=\color{black},     % string literal style
	tabsize=2,	                   % sets default tabsize to 2 spaces
	literate=
	*{:}{{{\color{black}{:}}}}{1}
	{,}{{{\color{black}{,}}}}{1}
	{\{}{{{\color{black}{\{}}}}{1}
	{\}}{{{\color{black}{\}}}}}{1}
	{[}{{{\color{black}{[}}}}{1}
	{]}{{{\color{black}{]}}}}{1},
}


\lstset{ %
	language=C++,                % choose the language of the code
	basicstyle=\footnotesize,       % the size of the fonts that are used for the code
	numbers=left,                   % where to put the line-numbers
	numberstyle=\footnotesize,      % the size of the fonts that are used for the line-numbers
	stepnumber=1,                   % the step between two line-numbers. If it is 1 each line will be numbered
	numbersep=5pt,                  % how far the line-numbers are from the code
	backgroundcolor=\color{white},  % choose the background color. You must add \usepackage{color}
	showspaces=false,               % show spaces adding particular underscores
	showstringspaces=false,         % underline spaces within strings
	showtabs=false,                 % show tabs within strings adding particular underscores
	frame=single,           % adds a frame around the code
	tabsize=2,          % sets default tabsize to 2 spaces
	captionpos=b,           % sets the caption-position to bottom
	breaklines=true,        % sets automatic line breaking
	breakatwhitespace=false,    % sets if automatic breaks should only happen at whitespace
	escapeinside={\%*}{*)}          % if you want to add a comment within your code
}



\newcommand{\HRule}{\rule{\linewidth}{0.5mm}}

\begin{document}
	\begin{titlepage}
		\begin{center}
			
			% Top 
			\includegraphics[width=0.45\textwidth]{img/unipi.png}~\\[2.5cm]
			
			
			% Title
			\HRule \\[0.4cm]
			{ \LARGE 
				\textbf{SpamDetector}\\[0.4cm]
				{Project for Data Mining and Machine Learning}\\[0.4cm]
			}
			\HRule \\[1.5cm]
			
			
			
			% Author
			{ \large
				Francesco Campilongo \\[0.1cm]
			}
			
			\vfill
			
		\end{center}
	\end{titlepage}

\tableofcontents
\chapter{Introduction}
The purpose of this documentation is to prove the machine learning algorithms utility for the detection of spam, useless messages, into dataset of emails.

\noindent This is a basic text classification problem, and so the algorithm used are very known in the lecture.

\noindent The project is carried out on python.
\chapter{Datasets}
The datasets used for this project are two:
\begin{itemize}
	\item E-mail dataset
	\item SMS dataset
\end{itemize}
Those two datasets where found on Kaggle.com both of them in a .csv format.
\subsection{Data Pre-Processing}
The datasets found were already pretty good for the type of result this project wants to achieve.

\noindent The E-mail dataset is characterize from three column a "label", "text" and "label\_num", the second column contains the actual message, the first and third column are the same, but in the first one there are the actual words "ham", for the messages which are not classified as spam, and "spam",  instead in the third column there is there are "0" for the ham messages and "1" for the spam messages;

\noindent The first column has been discharged since not useful for the execution of the classification algorithm.

\noindent The SMS file had need a little more work to obtain a dataset usable, because the it has five columns named "v1", "v2", "", "" and "", the first contains the words "ham", for the messages which are not classified as spam and "spam"; the second column contains the actual messages and the last three columns are basically blank, so in order to obtain a good dataset to work with the last three columns has been discharged the first two are taken in consideration, but in the first one the "ham" word is changed with a "0" and the "spam" word with a "1", in order to have the same kind of datasets between the E-mail and the SMSs.
\chapter{Design}
\chapter{Use Case Diagram}
\chapter{UML Class Analysis}
\chapter{Implementation and Tests}
\chapter{Results}
\end{document}